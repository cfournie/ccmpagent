\section{Methodology}
We began by dividing the problem into 3 main units: the ART agent, learning,
and trust.  With four team members, one was assigned to each unit, with the
exception of the trust framework, where two were assigned.

Unit testing (using JUnit 4 and Eclipse) was performed to create high quality
code, and also to ease integration issues later on.  Source control (Subversion
with daily checkin email) was used to also ease integration problems, as
everyone ensured that they checked in early, and often.  ANT built scripts were
created to specify the various deliverables, including the final JARs and
Javadoc to be created.

An IDE with source analysis (Eclipse) was used.  This was helpful in pointing
out warnings, and debugging, and ensuring that all code was compliant with JDK
1.5, our target build environment.
    
Discussion was done in person at first, and the tools, environment
information, and configurations were then hosted on a team website. Google Wave,
a mailing list, email, and Google Talk were then used for communication.